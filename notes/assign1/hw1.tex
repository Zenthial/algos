\documentclass{article}

\usepackage{amsmath}
\usepackage{amssymb}

\newcommand {\R}{\mathbb{R}}
\newcommand {\N}{\mathbb{N}}

\title{%
  Analysis of Algorithms \\
  \large Homework 1 }

\author{Thomas Schollenberger (tss2344)}

\begin{document}
\maketitle

\section*{Problem \#1}
Rank the following functions by order of growth. Further, partition the list into equivalence classes such that
functions \(f(n)\) and \(g(n)\) are in the same class iff \(f(n) \in \theta(g(n))\).
\begin{center}
\begin{tabular}{|c|c|c|c|c|}
    \hline
    \(\ln{(\ln{(x)}})\) & \(n2^n\) & \(n^{\lg{(\lg{(n)})}}\) & \(\ln{n}\) & \(1\) \\
    \hline
    \((\lg{(n)})^{\lg{(n)}}\) & \(e^n\) & \(4^{\lg{(n)}}\) & \((n + 1)!\) & \(\sqrt{\lg{(n)}}\) \\
    \hline
    \(n^3\) & \((\lg{(n)})^2\) & \(\lg{n!}\) & \(2^{2^{n}}\) & \(\frac{3}{2}^n\) \\
    \hline
    \(2^{\sqrt{2\lg{(n)}}}\) & \(n\) & \(2^{n}\) & \(n\lg{(n)}\) & \(2^{2^{n+1}}\) \\
    \hline
    \(2^{\lg{(n)}}\) & \(\sqrt{2}^{\lg{(n)}}\) & \(n^2\) & \(n!\) & \((\lg{(n)})!\) \\
    \hline
\end{tabular}
\end{center}
Equivalence Classes, where \(a \in \R\):
\[[\theta(1)] = \{1\},\]
\[[\theta(\sqrt{\lg{(x)}})] = \{\sqrt{\lg{(n)}}\},\]
\[[\theta(\ln{(x)})] = \{\ln{(\ln{(n)})}, ln{(n)}\},\]
\[[\theta(\lg{(x)})] = \{n\lg{(n)}, \lg{(n!)}\},\]
\[[\theta(x)] = \{n\},\]
\[[\theta(a^{\sqrt{n}})] = \{2^{\sqrt{2\lg{(n)}}}\},\]
\[[\theta(x^2)] = \{n^2, (\lg{(n)})^2\},\]
\[[\theta(x^3)] = \{n^3\},\]
\[[\theta(a^{\lg{(n)}})] = \{2^{\lg{(n)}}, \sqrt{2}^{\lg{(n)}}, (\lg{(n)})^{\lg{(n)}}, 4^{\lg{(n)}}\},\]
\[[\theta(n^{\lg{(n)}})] = \{n^{\lg{(\lg{(n)})}}\},\]
\[[\theta(a^n)] = \{n2^n, e^n, 2^n, \frac{3}{2}^n\},\]
\[[\theta(x!)] = \{(\lg{(n)})!, n!, (n + 1)!\}\]
\[[\theta(a^{a^x})] = \{2^{2^n}, 2^{2^{n+1}}\},\]
And the order of the functions, from highest to lowest, are:
\[2^{2^{n+1}}\]
\[2^{2^n}\]
\[(n+1)!\]
\[n!\]
\[e^{n}\]
\[n\left(2\right)^{n}\]
\[2^{n}\]
\[\frac{3}{2}^n\]
\[n^{\lg{(\lg{(n)})}}\]
\[(\lg{(n)})^{\lg{(n)}}\]
\[4^{\lg{(n)}}\]
\[\lg{(n)})!\]
\[n^3\]
\[n^2\]
\[n\lg{(n)}\]
\[\lg{(n!)}\]
\[2^{\lg{(n)}}, n \left(\text{These are functionally equivalent}\right)\]
\[\sqrt{2}^{\lg{(n)}}\]
\[2^{\sqrt{2\lg{(n)}}}\]
\[\lg{(n)}^2\]
\[ln{(n)}\]
\[\sqrt{\lg{(n)}}\]
\[\ln{(\ln{(n)})}\]
\[1\]

\section*{Problem \#2}
Let: \[a, b, c, k \in \R \]
We have the following functions: \[ \sqrt[k]{x}, a^x, x^c, \log_b{x} \]
We can translate them into: \[ x^\frac{1}{k}, a^x, x^c, \frac{\log{x}}{\log{b}} \]
For all constants, \(\lim_{x\to\infty}\): \[
    x^\frac{1}{k} < x^c,
    \frac{\log{x}}{\log{b}} < x^c,
    x^c < a^x,
    x^\frac{1}{k} < \frac{\log{x}}{\log{b}}
\]
Which gives us: \[ x^\frac{1}{k} < \frac{\log{x}}{\log{b}} < x^c < a^x \]

\section*{Problem \#3}
For all three proofs, consider:
\[ f(x) = n \implies f(x) \in \mathcal{O}(g(x)) \]
\[ N, c \in \R, N, c = 1 \]
\[ |f(x)| \leq c|g(x)| \]
\subsection*{a. Prove the following:}
Using the class definition of Big O, prove that \[ n = \mathcal{O}(n^2) \]
Suppose:
\[x \geq N \]
\[N \leq x \implies x \leq x^2 \]
\[x \leq x^2 \implies |x| \leq c|x^2| \]
Thus:
\[n = \mathcal{O}(n^2) \]
\subsection*{b. Prove the following:}
Using the class definition of Big O, prove that \[ n^2 = \mathcal{O}(n^2) \]
Suppose:
\[1 \geq 1 \]
\[1 \leq 1 \implies x \leq x \]
\[x \leq x \implies |x^2| \leq 1|x^2| \]
Thus:
\[n^2 = \mathcal{O}(n^2) \]
\subsection*{c. Prove the following:}
Using the class definition of Big O, prove that \[ 3n^2 + 5n = \mathcal{O}(n^2) \]
Suppose:
\[3x^2 + 5x \leq 3x^2 + 5x \]
\[3x^2 + 5x \leq 3x^2 + 5x^2 \]
\[3x^2 + 5x \leq 8x^2 \]
Thus:
\[N = 8, C = 1\]
\[|3x^2 + 5x| \leq 1|8x^2|\]
\[3x^2 + 5x = \mathcal{O}(x^2)\]

\section*{Project \#4}
Big \(\mathcal{O}\) proofs

\subsection*{a. Given that \(\sum_{k=2}^n \frac{1}{k} \leq \ln{(n)} - \ln{(1)}\), using the class definition of \(\mathcal{O}\), prove that \(H_n \in \mathcal{O}(\ln{(n)})\)}


\section*{Problem \#5}
The smallest \{n\} that fib starts slowing down at is 30.

\section*{Problem \#6}
Consider the following recurrence:
\[f(0; a, b) = a\]
\[f(1; a, b) = b\]
\[f(n; a, b) = f(n - 1; b, a + b)\]
\subsection*{a. Prove using mathematical induction that for any \(n \in \N\) if \(n > 1\) then \(f(n; a, b) = f(n - 1; a, b) + f(n - 2; a, b)\)}
% \[f(n; a, b) = f(n - 1; a, b) + f(n - 2; a, b)\]
Observe that, for \(n = 2\):
\begin{align}
f(2; a, b) &= f(1; b, a + b)\\
&= a + b\\
&= f(1; a, b) + f(0; a, b)\\
&= f(2 - 1; a, b) + f(2 - 2; a, b)
\end{align}

Assume that \(f(k; a, b) = f(k - 1; a, b) + f(k - 2; a, b)\) is true, where \(0 \leq k < n\), then:
\begin{align}
f(k + 1; a, b) &= f(k; b, a + b)\\
&= f(k - 1; b, a + b) + f(k - 2; b, a + b)\\
&= f(k; a, b) + f(k - 1; a, b)
\end{align}

Which proves f(n; a, b) = f(n - 1; a, b) + f(n - 2; a, b)

\section*{Problem \#7}
Exceeds maximum recursion depth before slowing down at all

\end{document}