\documentclass{article}

\usepackage{amsmath}
\usepackage{amssymb}

\newcommand {\R}{\mathbb{R}}
\newcommand {\N}{\mathbb{N}}

\title{%
  Analysis of Algorithms \\
  \large Homework 1 }

\author{Thomas Schollenberger (tss2344)}

\begin{document}
\maketitle

\section*{Problem \#1}
Rank the following functions by order of growth. Further, partition the list into equivalence classes such that
functions \(f(n)\) and \(g(n)\) are in the same class iff \(f(n) \in \theta(g(n))\).
\begin{center}
\begin{tabular}{|c|c|c|c|c|}
    \hline
    \(\ln{(\ln{(x)}})\) & \(n2^n\) & \(n^{\lg{(\lg{(n)})}}\) & \(\ln{n}\) & \(1\) \\
    \hline
    \((\lg{(n)})^{\lg{(n)}}\) & \(e^n\) & \(4^{\lg{(n)}}\) & \((n + 1)!\) & \(\sqrt{\lg{(n)}}\) \\
    \hline
    \(n^3\) & \((\lg{(n)})^2\) & \(\lg{n!}\) & \(2^{2^{n}}\) & \(\frac{3}{2}^n\) \\
    \hline
    \(2^{\sqrt{2\lg{(n)}}}\) & \(n\) & \(2^{n}\) & \(n\lg{(n)}\) & \(2^{2^{n+1}}\) \\
    \hline
    \(2^{\lg{(n)}}\) & \(\sqrt{2}^{\lg{(n)}}\) & \(n^2\) & \(n!\) & \((\lg{(n)})!\) \\
    \hline
\end{tabular}
\end{center}
Equivalence Classes, where \(a \in \R\):
\[[\theta(1)] = \{1\},\]
\[[\theta(\sqrt{\lg{(x)}})] = \{\sqrt{\lg{(n)}}\},\]
\[[\theta(\ln{(x)})] = \{\ln{(\ln{(n)})}, ln{(n)}\},\]
\[[\theta(\lg{(x)})] = \{n\lg{(n)}, \lg{(n!)}\},\]
\[[\theta(x)] = \{n\},\]
\[[\theta(a^{\sqrt{n}})] = \{2^{\sqrt{2\lg{(n)}}}\},\]
\[[\theta(x^2)] = \{n^2, (\lg{(n)})^2\},\]
\[[\theta(x^3)] = \{n^3\},\]
\[[\theta(a^{\lg{(n)}})] = \{2^{\lg{(n)}}, \sqrt{2}^{\lg{(n)}}, (\lg{(n)})^{\lg{(n)}}, 4^{\lg{(n)}}\},\]
\[[\theta(n^{\lg{(n)}})] = \{n^{\lg{(\lg{(n)})}}\},\]
\[[\theta(a^n)] = \{n2^n, e^n, 2^n, \frac{3}{2}^n\},\]
\[[\theta(a^{a^x})] = \{2^{2^n}, 2^{2^{n+1}}\},\]
\[[\theta(x!)] = \{n!, (\lg{(n)})!, (n + 1)!\}\]
Where the equivalence classes are ordered by growth, from \(\theta(1)\), being the lowest growth, to \(\theta(x!)\), being the highest growth.

\section*{Problem \#2}
Let: \[a, b, c, k \in \R \]
We have the following functions: \[ \sqrt[k]{x}, a^x, x^c, \log_b{x} \]
We can translate them into: \[ x^\frac{1}{k}, a^x, x^c, \frac{\log{x}}{\log{b}} \]
For all constants, \(\lim_{x\to\infty}\): \[
    x^\frac{1}{k} < x^c,
    \frac{\log{x}}{\log{b}} < x^c,
    x^c < a^x,
    x^\frac{1}{k} < \frac{\log{x}}{\log{b}}
\]
Which gives us: \[ x^\frac{1}{k} < \frac{\log{x}}{\log{b}} < x^c < a^x \]

\section*{Problem \#3}
For all three proofs, consider:
\[ f(x) = n \implies f(x) \in \mathcal{O}(g(x)) \]
\[ N, c \in \R, N, c = 1 \]
\[ |f(x)| \leq c|g(x)| \]
\subsection*{a. Prove the following:}
Using the class definition of Big O, prove that \[ n = \mathcal{O}(n^2) \]
Suppose:
\[x \geq N \]
\[N \leq x \implies x \leq x^2 \]
\[x \leq x^2 \implies |x| \leq c|x^2| \]
Thus:
\[n = \mathcal{O}(n^2) \]
\subsection*{b. Prove the following:}
Using the class definition of Big O, prove that \[ n^2 = \mathcal{O}(n^2) \]
Suppose:
\[x \geq N \]
\[N \leq x \implies x + x \leq x^2 + x \]
\[x + x \leq x^2 + x \implies |x^2| \leq |x^3| \]
Remember:
\[\mathcal{O}(x^2) \subset \mathcal{O}(x^3) \]
Thus, if:
\[n^2 = \mathcal{O}(n^3) \]
Then:
\[n^2 = \mathcal{O}(n^2) \]
\subsection*{c. Prove the following:}
Using the class definition of Big O, prove that \[ 3n^2 + 5n = \mathcal{O}(n^2) \]
Suppose:
\[3x^2 + 5x \leq 3x^2 + 5x \]
\[3x^2 + 5x \leq 3x^2 + 5x^2 \]
\[3x^2 + 5x \leq 8x^2 \]
Thus:
\[N = 8, C = 1\]
\[|3x^2 + 5x| \leq 1|8x^2|\]
\[3x^2 + 5x = \mathcal{O}(x^2)\]

\section*{Project \#4}
Big \(\mathcal{O}\) proofs

\subsection*{a. Given that \(\sum_{k=2}^n \frac{1}{k} \leq \ln{(n)} - \ln{(1)}\), using the class definition of \(\mathcal{O}\), prove that \(H_n \in \mathcal{O}(\ln{(n)})\)}



\section*{Problem \#6}
Consider the following recurrence:
\[f(0; a, b) = a\]
\[f(1; a, b) = b\]
\[f(n; a, b) = f(n - 1; b, a + b)\]
\subsection*{a. Prove using mathematical induction that for any \(n \in \N\) if \(n > 1\) then}
\[f(n; a, b) = f(n - 1; a, b) + f(n - 2; a, b)\]
Base Case, \(n = 2\):
\[f(n; a, b) = f(n - 1; a, b) + f(n - 2; a, b)\]
Step, \(n + 1\):
\[f(n + 1; a, b) \implies (n+1) + f(n; a, b) = f(n - 1; a, b) + f(n - 2; a, b) + (n + 1)\]

\end{document}