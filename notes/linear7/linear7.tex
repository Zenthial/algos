\documentclass{report}

\usepackage{amsfonts}
\usepackage{amsmath}
\usepackage{amssymb}
\usepackage[T1]{fontenc}
\usepackage{color}


\definecolor{light}{rgb}{0.3, 0.3, 0.3}
\def\light#1{{\color{light}#1}}

\input{preamble}
\input{macros}
\input{letterfonts}

\title{\Large{Analysis of Algorithms} \\ \huge{Homework 6}}

\author{\Large{Thomas Schollenberger (tss2344)}}
% \date{}

\begin{document}

\maketitle

\qs{1}{
    Verfify that \(\{u_1, u_2\}\) is an orthogonal set and find the projection of \(y\) onto span\(\{u_1, u_2\}\).
}

\[
  u_1 =
  \left[ {\begin{array}{c}
    -4 \\
    -1 \\
    1 \\
  \end{array} } \right]\text{,}
  u_2 =
  \left[ {\begin{array}{c}
    0 \\
    1 \\
    1 \\
  \end{array} } \right]\text{,}
  y =
  \left[ {\begin{array}{c}
    6 \\
    4 \\
    1 \\
  \end{array} } \right]
\]

\pf{Answer to 1}{
    \(u_1\left(u_2\right) = -4(0) + -1(1) + 1(1) = 0\) \\
    \[
        A =
        \left[ {\begin{array}{cc}
            -4 & 0 \\
            -1 & 1 \\
            1 & 1 \\
        \end{array} } \right]
    \] \\

    \(A^TA = \left[ {\begin{array}{cc}
        18 & 0 \\
        0 & 2 \\
    \end{array} } \right]\) \\

    \(A^Ty = \left[ {\begin{array}{c}
        21 \\
        5 \\
    \end{array} } \right]\) \\

    \begin{align*}
        &\left[ {\begin{array}{ccc}
            18 & 0 & 21 \\
            0 & 2 & 5 \\
        \end{array} } \right] \\
        &\left[ {\begin{array}{ccc}
            1 & 0 & \frac{7}{6} \\
            0 & 1 & \frac{5}{2} \\
        \end{array} } \right] \\
        &= \left[ {\begin{array}{c}
            \frac{7}{6} \\
            \frac{5}{2} \\
        \end{array} } \right] \\
    \end{align*}

    \[
        y_{proj} = A\left[ {\begin{array}{c}
            \frac{7}{6} \\
            \frac{5}{2} \\
        \end{array} } \right] = \frac{1}{3}\left[
            \begin{array}{c}
                14 \\
                4 \\
                11 \\
            \end{array}
        \right]  
    \]
}

\qs{2}{
    Let \(u_1, \dots , u_p\) be an orthogonal basis for a subspace \(W\) of \(\mathcal{R}^n\), and let \(T : \mathcal{R}^n \rightarrow \mathcal{R}^n
    \) be defined by \(T\)(x) = \(\text{proj}_W\) x. Show that \(T\) is a linear transformation
}

\pf{Answer to 2}{
    \(T\) is a linear transformation because it is a linear combination of the basis vectors. \\
    \[
        T\left( \sum_{i=1}^n c_i u_i \right) = \sum_{i=1}^n c_i \text{proj}_W u_i = \sum_{i=1}^n c_i u_i
    \]
}

\qs{3}{
    Use the Gram-Schmidt process to find an orthogonal basis for the column space of
}

\[
    A =
    \left[ {\begin{array}{ccc}
        1 & 3 & 5 \\
        -1 & 3 & 1 \\
        0 & 2 & 3 \\
        1 & 5 & 2 \\
        1 & 5 & 8 \\
    \end{array} } \right]
\]

\pf{Answer to 3}{
    \(u_1 = \left[ {\begin{array}{c}
        1 \\
        -1 \\
        0 \\
        1 \\
        1 \\
    \end{array} } \right]\),

    \(u_2 = \left[
        \begin{array}{c}
            3 \\
            3 \\
            2 \\
            5 \\
            5 \\
        \end{array}
    \right]\),

    \(u_3 = \left[
        \begin{array}{c}
            5 \\
            1 \\
            3 \\
            2 \\
            8 \\
        \end{array}
    \right]\)

    \(v_1 = u_1\)
    \(v_2 = u_2 - \frac{u_2\left(v_1\right)}{v_1\left(v_1\right)} v_1\)
    \(v_3 = u_3 - \frac{u_3\left(v_1\right)}{v_1\left(v_1\right)} v_1 - \frac{u_3\left(v_2\right)}{v_2\left(v_2\right)} v_2\)

    \(v_2 = \frac{1}{2}\left[
        \begin{array}{c}
            1 \\
            11 \\
            4 \\
            5 \\
            5 \\
        \end{array}
    \right]\)

    \(v_3 = \frac{1}{2}\left[
        \begin{array}{c}
            1 \\
            1 \\
            1 \\
            1 \\
            1 \\
        \end{array}
    \right]\)

    \(\left\{v_1, v_2, v_3\right\}\) is an orthogonal basis for the column space of \(A\)
}

\qs{4}{
    Let \(W\) be the span of the first two columns of the matrix from question \(3\), and let \(x \in W\) be defined
    below. Use your answer from question \(3\) to decompose \(x\) into a vector \(\hat{x} \in W\) and a vector \(z \in W^{\bot}\).
}

\[
    x = \left[
        \begin{array}{c}
            3 \\
            9 \\
            0 \\
            -9 \\
            3 \\
        \end{array}
    \right]
\]

\pf{Answer to 4}{
    \[
        x = \hat{x} + z
    \]

    \[
        \hat{x} = \frac{1}{2}\left[
            \begin{array}{c}
                1 \\
                11 \\
                4 \\
                5 \\
                5 \\
            \end{array}
        \right]
    \]

    \[
        z = \frac{1}{2}\left[
            \begin{array}{c}
                1 \\
                1 \\
                1 \\
                1 \\
                1 \\
            \end{array}
        \right]
    \]
}

\end{document}