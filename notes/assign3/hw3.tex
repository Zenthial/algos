\documentclass{article}

\usepackage{amsmath}
\usepackage{amssymb}
\usepackage[shortlabels]{enumitem}

\newcommand {\R}{\mathbb{R}}
\newcommand {\N}{\mathbb{N}}

\title{%
  Analysis of Algorithms \\
  \large Homework 3 }

\author{Thomas Schollenberger (tss2344)}

\begin{document}
    \maketitle
    \section*{Problem \#2}
    \begin{subequations}
        \begin{align}
            search([1,2,3], 2) &= searchHelp([1,2,3], 2, 0, 2) \\
            \hat{m} &= 0 + \lfloor(2-0)/2\rfloor \\
            \hat{m} &= 2 \\
            v &< a[\hat{m}] \\
            &= searchHelp([1,2,3], 2, 0, 2)
        \end{align}
    \end{subequations}
    Which results in an infinite loop, as the final searchHelp call is identical to the initial.
    This is because the \(\hat{m}\) calculation is returning 2, the length of the array minus one, 
    but the binary search operation is not incrementing or decrementing \(\hat{m}\) when recursively calling it.

    \section*{Problem \#3}
    \subsection*{a. Strassen's Algorithm computation:}
    \[
        \begin{pmatrix}
            1 & 3 \\
            7 & 5
        \end{pmatrix}
        \begin{pmatrix}
            6 & 8 \\
            4 & 2
        \end{pmatrix}
    \]
    \begin{subequations}
        \begin{align}
        M_1 &= (1 + 5)(6 + 2) \\
        M_2 &= (7 + 5)(6) \\
        M_3 &= (1)(8 - 2) \\
        M_4 &= (5)(4 - 6) \\
        M_5 &= (1 + 3)(2) \\
        M_6 &= (7 - 1)(6 + 8) \\
        M_7 &= (3 - 5)(4 + 2)
        \end{align}
    \end{subequations}
    \[
      \begin{bmatrix}
        C_{11} & C_{12} \\
        C_{21} & C_{22}
      \end{bmatrix}
      =
      \begin{bmatrix}
        M_1 + M_4 - M_5 + M_7 & M_3 + M_5 \\
        M_2 + M_4 & M_1 - M_2 + M_3 + M_6
      \end{bmatrix}
    \]
    \[
      \begin{bmatrix}
        C_{11} & C_{12} \\
        C_{21} & C_{22}
      \end{bmatrix}
      =
      \begin{bmatrix}
        18 & 14 \\
        62 & 66
      \end{bmatrix}
    \]
    \[
      C
      =
      \begin{bmatrix}
        18 & 14 \\
        62 & 66
      \end{bmatrix}
    \]
    \subsection*{b. Strassen's Algorithm pseudo-code:}
    \[
        \text{strassen}(A, B) = 
        \begin{cases}
            \text{A x B}, & \text{if a.rows = 1} \\
            \begin{pmatrix}
                C_{11} & C_{12} \\
                C_{21} & C_{22}
            \end{pmatrix}, & \text{otherwise }
        \end{cases}
    \]
    \begin{align*}
        \text{where}
        \begin{pmatrix}
            A_{11} & A_{12} \\
            A_{21} & A_{22}
        \end{pmatrix}
        &= A \\
        \begin{pmatrix}
            B_{11} & B_{12} \\
            B_{21} & B_{22}
        \end{pmatrix}
        &= B \\
        P_1 = \text{strassen}&(A_{11}, (B_{12} - B_{22})) \\
        P_2 = \text{strassen}&((A_{11} + A_{12}), B_{22}) \\
        P_3 = \text{strassen}&((A_{21} + A_{22}), B_{11}) \\
        P_4 = \text{strassen}&(A_{22}, (B_{21} - B_{11})) \\
        P_5 = \text{strassen}&((A_{11} + A_{22}), (B_{11} + B_{22})) \\
        P_6 = \text{strassen}&((A_{12} - A_{22}), (B_{21} + B_{22})) \\
        P_7 = \text{strassen}&((A_{11} - A_{21}), (B_{11} + B_{12})) \\
        C_{11} = P_5 &+ P_4 - P_2 + P_6 \\
        C_{12} = P_1 &+ P_2 \\
        C_{21} = P_3 &+ P_4 \\
        C_{22} = P_5 &+ P_1 - P_3 - P_7
    \end{align*}
    \subsection*{c. Verify that \(C_{2,1} = A_{21}B_{11} + A_{22}B_{21}\)}
    \begin{align*}
        C_{2,1} &= P_3 + P_4 \\
        &= (A_{21} + A_{22})B_{11} + A_{22}(B_{21} - B_{11})\\
        &= A_{21}B_{11} + A_{22}B_{11} + A_{22}B_{21} - A{22}B_{11} \\
        &= A_{21}B_{11} + A_{22}B_{21}
    \end{align*}
    \subsection*{d. Verify that \(C_{2,2} = A_{21}B_{12} + A_{22}B_{22}\)}
    \[C_{22} = P_5 + P_1 - P_3 - P_7\]
    \[= (A_{11} + A_{22})(B_{11} + B_{22}) + A_{11}(B_{12} - B_{22}) - (A_{21} + A_{22})B_{11} - (A_{11} - A_{21})(B_{11} + B_{12})\]
    \[= (A_{11} + A_{22})(B_{11} + B_{22}) + A_{11}B_{12} - A_{11}B_{22} - A_{21}B_{11} + A_{22}B_{11} - (A_{11} - A_{21})(B_{11} + B_{12})\]
    \[= A_{11}B_{11} + A_{11}B_{22} + A_{22}B_{11} + A_{22}B_{22} + A_{11}B_{12} - A_{11}B_{22} - A_{21}B_{11} + A_{22}B_{11} - (A_{11} - A_{21})(B_{11} + B_{12})\]
    \[= A_{11}B_{11} + A_{22}B_{11} + A_{22}B_{22} + A_{11}B_{12} - A_{21}B_{11} + A_{22}B_{11} - A_{11}B_{11} - A{21}B_{11} - A_{11}B_{12} - A_{21}B_{12}\]
    \[= A_{22}B_{11} + A_{22}B_{22} - A_{21}B_{11} - A_{22}B_{11} - A{21}B_{11} + A_{21}B_{12}\]
    \[= A_{22}B_{22} - A_{21}B_{11} + A{21}B_{11} + A_{21}B_{12}\]
    \[= A_{22}B_{22} + A_{21}B_{12}\]
    \subsection*{e. Solve this recurrence, assuming \(n = 2^m\)}
    \begin{align*}
        T(1) = 1 \\
        T(n) = 7T(\frac{n}{2}) + \frac{9}{2}n^2
    \end{align*}
    \begin{subequations}
        \begin{align}
            T(n) &= 7T(\frac{n}{2} + \frac{9}{2}n^2) \\
            &= 7[7T(\frac{\frac{n}{2}}{2}) + \frac{9}{2}\frac{n}{2}^2] + \frac{9}{2}n^2 \\
            &= 7[7[7T(\frac{\frac{\frac{n}{2}}{2}}{2}) + \frac{9}{2}\frac{n}{4}^2] + \frac{9}{2}\frac{n}{2}^2] + \frac{9}{2}n^2 \\
            &= 7[7[7T(2^{m-3}) + \frac{9}{2}(2^{m-2})^2] + \frac{9}{2}(2^{m-1})^2] + \frac{9}{2}(2^{m})^2 \\
            &= 7^3T(2^{m-3}) + 7^2\frac{9}{2}(2^{m-2})^2 + 7\frac{9}{2}(2^{m-1})^2 + \frac{9}{2}(2^{m})^2 \\
            &= 7^kT(2^{m-k}) + 7^{k-1}\frac{9}{2}(2^{m-(k-1)})^2 + 7^{k-2}\frac{9}{2}(2^{m-(k-2)})^2 + 7^{k-3}\frac{9}{2}(2^{m-(k-3)})^2 \\
            &= 7^kT(2^{m-k}) + \sum_{i=0}^{k-1} 7^{i}\frac{9}{2}(2^{m-(i)})^2 \\
            &= 7^{\lg{n}} - 6n^2
        \end{align}
    \end{subequations}
    \subsection*{f. Modify Strassen's algorithm to work for a \(n\) x \(n\), where \(n\) is not an exact power of 2}
    To achieve this, we would need to add zeros to the \(n\) x \(n\) matrix so it could become a \(2n\) x \(2n\) matrix.
    From there, we would be performing 7 operations on a 2n matrix, which would make it run in \(\theta(2n^{\lg{(7)}})\), which is equivalent to \(\theta(n^{\lg{(7)}})\)
    \subsection*{g. Show how to multiply complex numbers \(a + bi\) and \(c + di\) using only three multiplications of real numbers.
      The algorithm should take the real numbers a, b, c, and d as input and produce the real component \(ac - bd\)
      and the imaginary component \(ad + bc\) separately.
    }
    \[
    \text{mult(a, b, c, d) = (real, imaginary)\text{, where:}}  
    \]
    \begin{align*}
      mul_{1} &= a(c) \\
      mul_{2} &= b(d) \\
      mul_{3} &= (a+b)(c+d) \\
      \text{real} &= mul_{1} - {mul_2} \\
      \text{imaginary} &= mul_{3} - {mul_1} - mul_{2}
    \end{align*}
    \section*{Problem \#4}
    Consider the recurrence \(T(1) = 0\), \(T(n) = T(\lfloor n/2 \rfloor) + T(\lceil n/2 \rceil) + n\).
    \subsection*{a. Let \(D(n) = T(n + 1) - T(n)\). It is a fact that \(D(1) = 2\), \(D(n) = D(\lfloor n/2 \rfloor) + 1\). Prove using the strong
    form of induction that for any \(n \in N\), if \(n \geq 1\) then \(D(n) = \lfloor \lg{(n)}\rfloor + 2\).}
    \[\text{Proof by Strong Induction:}\]
    \[\text{Base Cases:}\]
    \[D(n) = T(n + 1) - T(n)\]
    \[D(1) = 2\]
    \[D(n) = D(\lfloor n/2 \rfloor) + 1\]
    This means that \(\lfloor n \rfloor = 2^k\), \(D(n/2) = \lfloor\lg{(n)}\rfloor + D(1)\), thus:
    \begin{align*}
      D(n) &= D(\lfloor n/2 \rfloor) + 1 \\
      D(n) &= \lfloor\lg{(n)}\rfloor + 1 + 1 \\
      D(n) &= \lfloor\lg{(n)}\rfloor + 2
    \end{align*}
    For all \(n \in \N\), where \(n \geq 1\)
    \subsection*{b. Then prove that \(T(n) - T(1) = \sum_{k=1}^{n-1} D(k)\), and show that an immediate consequence is that \(T(n) = \sum_{k=1}^{n-1} (\lfloor\lg{(n)}\rfloor + 2)\)}
    We know that
    \[D(n) = \lfloor \lg{(n)} \rfloor + 2\]
    \[D(n) = T(n + 1) - T(n)\]
    \[D(n - 1) = T(n) - T(n - 1)\]
    Furthermore
    \[T(n) - T(1) = T(n) - 0 = T(n)\]
    \[T(n) = \lfloor n\lg{(n)} \rfloor + n\]
    Which follows that
    \[\lfloor n\lg{(n)} \rfloor + n - T(1) = \sum_{k=1}^{n-1} D(k)\]
    \[\lfloor n\lg{(n)} \rfloor + n = \sum_{k=1}^{n-1} (\lfloor\lg{(n)}\rfloor + 2)\]
    \subsection*{c. Now show that \(T(n) = \sum_{k=1}^{n-1} (\lfloor\lg{(n)}\rfloor + 2)\), implies that \(T(n) = O(n\log{(n)})\)}
    Recall that
    \[\sum_{k=1}^{n} \lg{(n)} = O(n\lg{(n)})\]
    Which implies that
    \[T(n) = \sum_{k=1}^{n-1} (\lfloor\lg{(n)}\rfloor + 2)\]
    \[T(n) = O(n\log{(n)})\]

\end{document}