\documentclass{report}

\usepackage{amsfonts}
\usepackage{amsmath}
\usepackage{amssymb}
\usepackage[T1]{fontenc}
\usepackage{color}


\definecolor{light}{rgb}{0.3, 0.3, 0.3}
\def\light#1{{\color{light}#1}}

\input{preamble}
\input{macros}
\input{letterfonts}

\title{\Large{Linear Algebra} \\ \huge{Homework 7}}

\author{\Large{Thomas Schollenberger (tss2344)}}
% \date{}

\begin{document}

\maketitle

\[
  A =
  \left[ {\begin{array}{ccc}
    4 & -1 & 6 \\
    2 & 1 & 6 \\
    2 & -1 & 8 \\
  \end{array} } \right]
\]

\qs{1}{
    Begin by computing the characteristic polynomial of \(A\) (i.e. compute det\((A-\lambda I))\). Use your polynomial
    to find the eigenvalues of \(A\). (Note, you may use a computer to factor the characteristic polynomial).
}
\pf{Answer to 1}{
    \(-\lambda^3 + 13\lambda^2 - 40\lambda + 36\) \\
    \(-\lambda^3 + 13\lambda^2 - 40\lambda + 36 = 0\) \\
    \(\lambda = 2, 9\) \\
    Eigenvalues are \(2\) and \(9\).
}

\qs{2}{
    For each eigenvalue from part \(1\) compute a basis for the associated eigenspace. i.e., Compute a basis
    for each null\((A - \lambda I)\).
}
\pf{Answer to 2}{
    \begin{align}
        &\text{null}(
            \left[ {\begin{array}{ccc}
                4 & -1 & 6 \\
                2 & 1 & 6 \\
                2 & -1 & 8 \\
            \end{array} } \right]
          -
            \left[ {\begin{array}{ccc}
                2 & 0 & 0 \\
                0 & 2 & 0 \\
                0 & 0 & 2 \\
            \end{array} } \right]
        ) \\
        &\text{null}(
            \left[ {\begin{array}{ccc}
                2 & -1 & 6 \\
                2 & -1 & 6 \\
                2 & -1 & 6 \\
            \end{array} } \right]
        ) \\
        & \{ \left[ {\begin{array}{c}
            \frac{1}{2} \\
            1 \\
            0 \\
        \end{array} } \right],
        \left[ {\begin{array}{c}
            -3 \\
            0 \\
            1 \\
        \end{array} } \right]
        \}
    \end{align}
    and,
    \begin{align}
        &\text{null}(
            \left[ {\begin{array}{ccc}
                4 & -1 & 6 \\
                2 & 1 & 6 \\
                2 & -1 & 8 \\
            \end{array} } \right]
          -
            \left[ {\begin{array}{ccc}
                9 & 0 & 0 \\
                0 & 9 & 0 \\
                0 & 0 & 9 \\
            \end{array} } \right]
        ) \\
        &\text{null}(
            \left[ {\begin{array}{ccc}
                -5 & -1 & 6 \\
                2 & -8 & 6 \\
                2 & -1 & -1 \\
            \end{array} } \right]
        ) \\
        & \{ \left[ {\begin{array}{c}
            1 \\
            1 \\
            1 \\
        \end{array} } \right]
        \}
    \end{align}
}

\qs{3}{
    Use your answers from parts \(1\) and \(2\) to find a diagonal matrix \(D\) and an invertible matrix \(P\) such that \(A = PDP^{-1}\)
}

\pf{Answer to 3}{
    \[
        D =
        \left[ {\begin{array}{ccc}
          2 & 0 & 0\\
          0 & 2 & 0 \\
          0 & 0 & 9 \\
        \end{array} } \right]    
    \]
    \[
        P =
        \left[ {\begin{array}{ccc}
            1 & -3 & 1 \\
            2 & 0 & 1 \\
            0 & 1 & 1 \\
        \end{array} } \right]
    \]
}

\qs{4}{
    Check the matrices you found in part \(3\) by computing \(AP\) and \(PD\). They should be the same.
}

\pf{Answer to 4}{
    \begin{align}
        &\left[ {\begin{array}{ccc}
            4 & -1 & 6 \\
            2 & 1 & 6 \\
            2 & -1 & 8 \\
        \end{array} } \right]
        \left[ {\begin{array}{ccc}
            1 & -3 & 1 \\
            2 & 0 & 1 \\
            0 & 1 & 1 \\
        \end{array} } \right] \\
        & \left[
            \begin{array}{ccc}
                -6 & 2 & 9 \\
                0 & 4 & 9 \\
                2 & 0 & 9 \\
            \end{array}
        \right]
    \end{align}
    and,
    \begin{align}
        &\left[ {\begin{array}{ccc}
            1 & -3 & 1 \\
            2 & 0 & 1 \\
            0 & 1 & 1 \\
        \end{array} } \right]
        \left[ {\begin{array}{ccc}
            2 & 0 & 0\\
            0 & 2 & 0 \\
            0 & 0 & 9 \\
        \end{array} } \right] \\
        & \left[
            \begin{array}{ccc}
                -6 & 2 & 9 \\
                0 & 4 & 9 \\
                2 & 0 & 9 \\
            \end{array}
        \right]
    \end{align}
    which shows that \(P\) and \(D\) are correct.
}

\qs{5}{
    What is \(D^n\)?
}

\pf{Answer to 5}{
    \[
        D^n =
        \left[ {\begin{array}{ccc}
          2^n & 0 & 0\\
          0 & 2^n & 0 \\
          0 & 0 & 9^n \\
        \end{array} } \right]
    \]
}

\qs{6}{
    Use your answer from parts \(3\) and \(5\) to compute a formula for \(A^n\).
}

\pf{Answer to 6}{
    \[
        A^n =
        \left[ {\begin{array}{ccc}
            1 & -3 & 1 \\
            2 & 0 & 1 \\
            0 & 1 & 1 \\
        \end{array} } \right]
        \left[ {\begin{array}{ccc}
          2^n & 0 & 0\\
          0 & 2^n & 0 \\
          0 & 0 & 9^n \\
        \end{array} } \right]
        \left[ {\begin{array}{ccc}
            1 & 2 & 0 \\
            -3 & 0 & 1 \\
            1 & 1 & 1 \\
        \end{array} } \right]^{-1}
    \]
}

\end{document}